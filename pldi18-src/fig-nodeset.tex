\begin{figure}
\begin{lstlisting}[
    %frame=single,
    basicstyle=\scriptsize,%\footnotesize,
    keepspaces=true,
    numbers=left,
    %numbersep=1em,
    xleftmargin=2em,
    numberstyle=\tiny,
    emph={
      %% assert, assume,
      %% if, then else,
      %% while, do,
      %% sort, relation, function,
      %% axiom,
      %% insert,
    },
    emphstyle={\bfseries},
    mathescape=true,
  ]
module $\mnodeset$ {
  type t
  relation $\rmember$ : $\snode,\text{t}$
  relation $\rmajority$ : t
  function $\risize$ : t $ \to \sint$

  interpret t as array<$\sint$,$\snode$> $\label{line:nodeset-t}$
  interpret $\rmember(n,s)$ as $\exists i. 0 \leq i < \rlen(s) \land \rvalue(s,i,n)$ $\label{line:nodeset-member}$
  interpret $\rmajority(s)$ as $\risize(s) + \risize(s) > \risize(\rallnodes)$ $\label{line:nodeset-majority}$

  invariant $\iintersect$ = $\forall s_1, s_2. \; \rmajority(s_1) \land \rmajority(s_2) \,\to$
                            $\exists n. \; \rmember(n, s_1) \land \rmember(n, s_2)$ $\label{line:nodeset-intersect}$

  procedure $\pemptyset()$ returns s:t {
    ensures $\forall n. \; \neg\rmember(n,\text{s})$
    s := array.empty()
  }
  procedure $\padd(\text{s}_1 : \text{t} ,\, \text{n} : \snode)$ returns $\text{s}_2 : \text{t}$ {
    ensures $\forall n'. \; \rmember(n', \text{s}_2) \leftrightarrow (\rmember(n', \text{s}_1) \lor n' = n)$
    if $\rmember(\text{n}, \text{s}_1)$ then { $\text{s}_2$ := $\text{s}_1$ } else { s2 := array.append($\text{s}_1$, n) }
  }
  procedure $\pinit()$ {
    $\risize$ := $\lambda x. 0$;
    for $0 \leq i < \rlen(\rallnodes)$ {
      invariant $\forall s_1,s_2. (\forall n. \neg (\rmember(n,s_1) \land \rmember(n, s_2))) \, \to$ $\label{line:nodeset-inv}$
          $\risize(s_1) + \risize(s_2) \leq \risize(\rallnodes)$
      $\risize$ := $\lambda x. \big( (\risize(x) + 1)$ if $\rmember(\text{array.get}(\rallnodes,i),x)$ else $\risize(x)\big)$
    }
  }
}
\end{lstlisting}
\caption{\label{fig:nodeset}$\mnodeset$ module for node sets, proving the majority intersection property.}
\end{figure}

\commentout{
  ghost relation $\rreachedset$ : t
  interpret t as array<$\sint$,$\snode$> with $\rreachedset$
  interpret $\rreachedset(s)$ as $\rlen(s) = \risize(s)$

  invariant $0 \leq i \leq \rlen(\rallnodes)$

}

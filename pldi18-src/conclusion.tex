\section{Conclusion}


\commentout{
At the implementation level, distributed protocols involve arithmetic,
unbounded sets of process and unbounded data structures.  For this
reason, we might expect that reasoning about these systems would
require the use of undecidable logics. We have seen, however, that by a
fairly simple modular decomposition, we can separate the proof into
lemmas in decidable fragments, which in turn can make the use of
automated provers more predictable and transparent. This results in
part from the fact that there are fairly rich theories that are
decidable and implemented in modern SMT solvers (particularly
Z3). Particularly, one needs to take care in separating arithmetic
reasoning from reasoning about parameterized systems, and to divide
the problem in a way that keeps functions and quantifier alternations
stratified. Useful tactics for this are to use an abstract protocol
model to prove system-level properties and to create abstract
datatypes.  Of perhaps independent interest is that the refinement
from abstract model to concurrent implementation can be accomplished
with traditional modular program reasoning methods, and without
establishing a simulation or using two-vocabulary formulas to specify
the abstract transition relation as in~\cite{IronFleet}.

There are, of course, still many problems to be solved, for example,
reducing the trusted code base and proving liveness. Our experience to
this point, however, indicates that robust proof automation can be
achieved without increasing the burden of annotation for engineers.
}

Modularity is well recognized as a key to scalability of systems.
This paper shows that modularity enables decidability of reasoning
about real implementations of distributed protocols.
%
% At the implementation level, distributed protocols involve arithmetic, unbounded sets of processes and unbounded data structures.
Such implementations involve arithmetic, unbounded sets of processes, and unbounded data structures.
For this reason, we might expect that reasoning about these systems
would require the use of undecidable logics. We have seen, however,
that by a fairly simple modular decomposition, we can separate the
proof into lemmas that reside in decidable fragments, which in turn
can make the use of automated provers more predictable and transparent.
%This results in part from the fact that there are
%fairly rich theories that are decidable and implemented in modern
%SMT solvers (particularly Z3). Particularly, one needs to take care
%in separating arithmetic reasoning from reasoning about
%parameterized systems, and to divide the problem in a way that keeps
%the us


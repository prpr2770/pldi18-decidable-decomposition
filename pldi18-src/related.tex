\section{Related Work}
\label{sec:related}

\paragraph{Fully Automatic Verification}
\begin{sloppypar}
Fully automatic
%push-button
verification of distributed protocols and systems is usually beyond reach because of
undecidability. Bounded checking is successfully used, e.g., in
Alloy~\cite{alloy} and TLA+~\cite{tla}, to check correctness of designs
up to certain numbers of nodes. This is useful, due to the
observation that most bugs occur with small numbers of nodes.  However
as observed by Amazon~\cite{amazon} and others it is hard to scale
these methods even for very few, e.g., $3$ nodes. Also many of the
interesting bugs occur in the implementation.
%One interesting approach to obtain decidability of fully automatic verification is via limiting the class of programs~\cite{vienna_book,konnov_smt_2015,konnov_short_2017,DBLP:conf/ershov/KonnovVW15}.
% ODED: moved to next paragraph
Another approach for automation is to use sound and incomplete procedures for deduction and invariant search for logics that combine quantifiers and set cardinalities~\cite{gleissenthall_cardinalities_2016,sally}.
However, distributed systems of the level of complexity we consider here (i.e., Raft, Multi-Paxos) are beyond the reach of these techniques.
Another direction, explored in~\cite{DBLP:journals/pacmpl/BakstGKJ17}, is to verify limited properties (e.g., for absence of deadlocks), using a sound but incomplete decidable check.
None of the state-of-the-art techniques for fully automatic verification can prove properties such as consistency for systems implementations.
Moreover, when automatic methods fail, the user is usually left without any solution.
\end{sloppypar}

\paragraph{Interactive Verification}
%Recent projects such as Verdi~\cite{Verdi} and IronFleet~\cite{IronFleet} demonstrate
The IronFleet~\cite{IronFleet} and Verdi~\cite{Verdi} projects recently demonstrated
that distributed systems can be proved all the way from design to implementation.
%
The DistAlgo project~\cite{DistAlgo,chand_formal_2016} develops programming methodologies for interactively verifying distributed systems using the TLA+ proof system~\cite{chaudhuri_tla+proof_2010}.
However, interactive verification requires tremendous human efforts and skills.
% and thus has limited success so far.
Our work can be considered as an attempt to understand how much automation is achievable using modularity.
We argue that invariants provide a reasonable way to interact with verification tools, since one does not need
to understand how decision procedures such as SMT work.
%In this work we express invariants in first-order logic.
%In the future, it may be possible to develop high-level specification approaches.
%This work considers invariants in first-order logic, and an interesting future direction is to develop specification approaches.
While this work considers invariant in first-order logic, it may be possible aplly its principles to richer specification approaches.

\paragraph{Decidable Reasoning about Distributed Protocols}
\begin{sloppypar}
PSync~\cite{dragoi_psync:_2016} is a DSL for distributed systems, with
decidable invariant checking in the $\mathbb{CL}$
logic~\cite{dragoi_logic-based_2014}.  Decidability is obtained by
restricting to the partially synchronous Heard-Of Model. A partially
synchronous model is also used in \cite{alberti_counting_2016}, which
develops a decidable fragment that allows some arithmetic with
function symbols and cardinality constraints.
%
The Threshold Automata formalism and the ByMC verification tool
\cite{vienna_book,konnov_smt_2015,konnov_short_2017,DBLP:conf/ershov/KonnovVW15}
allow to express a restricted class of distributed algorithms
operating in a partially synchronous communication mode. This
restriction allows decidability results based on cutoff theorems, for
both safety and liveness.
%
Recently, \cite{MSB17} presented a cutoff result for consensus
algorithms. This allows to verify, e.g., the core Paxos algorithm,
using a cutoff bound of 5 processes. However, this work is focused on
algorithms for the core consensus problem, and does not support
infinite-state per process, that is needed, e.g., to model Multi-Paxos
and SMR.
%
Compared to these works, our approach considers a more general setting
of asynchronous communication, and uses more restricted and mainstream
decidable fragments that are supported by existing theorem
provers. This is enabled by our use of modularity. Our approach of
applying modularity to obtain decidability will benefit and become
more powerful as more expressive decidable logics are developed and
supported by efficient solvers.
\end{sloppypar}

%% Compared to previous work on Ivy~\cite{oopsla17-epr,ken_fmcad16,Ivy},
%% this work is the first to produce verified implementation, exploiting
%% modularity in a new way, and also using the FAU
%% fragment~\cite{ge2009complete}, which is more expressive than EPR.

\oded{Can we remove the following now that we are de-anonymized?
Recently,~\cite{oopsla17-epr} showed how to verify abstract protocols
of the Paxos family in EPR, and \cite{ken_fmcad16} used it for cache
coherence. In this work we use EPR and FAU to reason about abstract
protocols but also go beyond the protocol level to verify an actual
efficient system implementation. The gap between the abstract protocol
and the system implementation presents a major technical challenge. We
bridge this gap by the novel use of modularity to obtain decidability,
as well as the use of the FAU fragment~\cite{ge2009complete}, which is
more expressive than EPR.}

\paragraph{Modularity in Verification}
The utility of modularity for simplified reasoning was already
recognized in the seminal works of Hoare and Dijkstra (e.g.,
\cite{Hoare:DataRepCorrectness}). Proof assistants such as
Isabelle/HOL~\cite{nipkow_isabelle/hol:_2002} and Coq~\cite{coq}
provide various modularity mechanisms. Deductive
verification engines such as Dafny~\cite{Dafny} and VCC~\cite{VCC} employ modularity to
simplify reasoning in a way that is similar to ours. The program logic
Disel~\cite{disel}
%uses protocols to allow
allows modular verification of
distributed systems, with rules somewhat similar to ours.
In~\cite{paxos_destructed_abstracted}, a series of transformations are
applied to obtain a verified implementation of Multi-Paxos from
single-decree Paxos in a compositional manner.
%
In this landscape, our chief novelties are the use of modularity for
decidability, and a methodology for modular, decidable reasoning about
distributed systems. We note that unlike Dafny, Ivy performs syntactic
checks to ensure that verification conditions are in decidable
logics. Thus, the user either receives an error message and corrects
the specification or can be assured that the verification problem is
solvable. Our evaluation demonstrates that our methodology is useful
for verifying complex distributed systems, and that it drastically
reduces the verification effort.
%
%\commentout{
%IVy, oopsla17-epr
%
%distalgo
%
%verdi
%
%ironfleet
%
%psync
%
%threshold automata
%
%ranjit oopsla17
%}
%
%\TODO{all the text below is copied as is from the oopsla17}
%
%\paragraph{Automated verification of distributed protocols}
%
%Here we review several works that developed techniques for automated
%verification of distributed protocols, and compare them with our
%approach.
%
%%% CL and PSync
%
%The Consensus Verification Logic
%$\mathbb{CL}$~\cite{dragoi_logic-based_2014} is a logic tailored to
%verify consensus algorithms in the partially synchronous Heard-Of
%Model~\cite{charron-bost_heard-model:_2009}, with a decidable fragment
%that can be used for verification. PSync~\cite{dragoi_psync:_2016} is
%a domain-specific programming language and runtime for developing formally
%verified implementations of consensus algorithms based on
%$\mathbb{CL}$ and the Heard-Of Model. Once the user provides inductive
%invariants and ranking functions in $\mathbb{CL}$, safety and liveness
%can be automatically verified. PSync's verified implementation of
%LastVoting (Paxos in the Heard-Of Model) is comparable in
%performance with state-of-the-art unverified systems.
%%% Compared to
%%% these works, our approach applies to a more general setting. The
%%% Heard-Of Model is partially synchronous, while our methodology
%%% applies to a general asynchronous setting.
%
%%% Viena
%
%Many interesting theoretical decidability results, as well as the ByMC
%verification tool, have been developed based on the formalism of
%Threshold Automata
%\cite{vienna_book,konnov_smt_2015,konnov_short_2017,DBLP:conf/ershov/KonnovVW15}. This
%formalism allows to express a restricted class of distributed
%algorithms operating in a partially synchronous communication
%mode. This restriction allows decidability results based on cutoff
%theorems, for both safety and liveness.
%%% However, the asynchronous
%%% Paxos algorithms and the variants considered in this work are beyond
%%% the reach of this formalism.
%
%\cite{alberti_counting_2016} present a decidable fragment of
%Presburger arithmetic with function symbols and cardinality constraint
%over interpreted sets.  Their work is motivated by applications to the
%verification of fault-tolerant distributed algorithms, and they
%demonstrate automatic safety verification of some fault-tolerant
%distributed algorithms expressed in a partially synchronous
%round-by-round model similar to PSync.
%
%\newcommand{\sharpie}{\#\Pi} $\sharpie$
%\cite{gleissenthall_cardinalities_2016} present a logic that combines
%reasoning about set cardinalities and universal quantifiers, along with an
%invariant synthesis method. The logic is not decidable, so a sound and
%incomplete reasoning method is used to check inductive
%invariants. Inductive invariants are automatically synthesized by
%method of Horn constraint solving. The technique is applied to
%automatically verify a collection of parameterized systems, including
%mutual exclusion, consensus, and garbage collection. However,
%Paxos-like algorithms are beyond the reach of this verification
%methodology since they require more complicated inductive invariants.
%
%\newcommand{\consl}{\mathit{ConsL}} Recently, \cite{MSB17} presented a
%cutoff result for consensus algorithms.  They define $\consl$, a
%domain specific language for consensus algorithms,
%whose semantics is based on the Heard-Of Model. $\consl$ admits a cutoff
%theorem, i.e., a parameterized algorithm expressed in $\consl$ is
%correct (for any number of processors) if and only if it is correct
%up to a some finite bounded number of processors (e.g., for Paxos the bound is 5) .
%%Thus, correctness of the parameterized algorithm can be proven by finite-state model checking.
%This theoretical result shows that for algorithms expressible in
%$\consl$, verification is decidable. However, $\consl$ is focused on
%algorithms for the core consensus problem, and does not support the
%infinite-state per process that is needed, e.g., to model Multi-Paxos
%and SMR.
%%% \TODO{can anyone check that the last paragraph is correct?}
%%% \TODO{maybe send them an email? I don't even understand how they model
%%%   Paxos, since the rounds are still infinite even with finite
%%%   processors.}
%
%The above mentioned works obtain automation (and some decidability) by
%restricting the programming model. We note that our approach takes a
%different path to decidability compared to these works. We axiomatize
%arithmetic, set cardinalities, and other higher-order concepts in an
%uninterpreted first-order abstraction. This is in contrast to the
%above works, in which these concepts are baked into specially designed
%logics and formalisms. Furthermore, we start with a Turing-complete
%modeling language and invariants with unrestricted quantifier
%alternation, and provide a methodology to reduce quantifier
%alternation to obtain decidability.  This allows us to employ a
%general-purpose decidable logic to verify asynchronous Paxos,
%Multi-Paxos, and their variants, which are beyond the reach of all of
%the above works.
%
%\paragraph{Deductive verification in undecidable logic}
%IronFleet~\cite{IronFleet} is a verified implementation of SMR, using
%the Dafny~\cite{dafny} program verifier, with verified safety and
%liveness properties. Compared to our work, this system implementation
%is considerably more detailed. The verification using Dafny employs Z3
%to check verification conditions expressed in undecidable logics that
%combine multiple theories and quantifier alternations. This leads to
%great difficulties due to the unpredictability of the solver.  To
%mitigate some of this unpredictability, IronFleet adopted a style they
%call \emph{invariant quantifier hiding}.  This style attempts to
%specify the invariants in a way that reduces the quantifiers that are
%explicitly exposed to the solver. Our work is motivated by the
%IronFleet experience and observations. The methodology we develop
%provides a more systematic treatment of quantifier alternations, and
%reduces the verification conditions to a decidable logic.
%
%\paragraph{Verification using interactive theorem provers}
%
%Recently, the Coq~\cite{coq} proof assistant has been used to develop
%verified implementations of systems, such as a file system
%\cite{fscq}, and shared memory data structures
%\cite{DBLP:conf/pldi/SergeyNB15}. Closer to our work is
%Verdi~\cite{DBLP:conf/pldi/WilcoxWPTWEA15}, which presents a verified
%implementation of an SMR system based on Raft~\cite{raft}, a
%Paxos-like algorithm.
%%Such efforts demonstrate that interactive
%%theorem provers allow a path to verified system
%%implementations. However, this
%This approach requires great effort, due
%to the manual process of the proof; developing a
%verified SMR system requires many months of work by verification
%experts, and proofs measure in thousands of lines.
%
%\cite{rahli_15th_2015, schiper_developing_2014} verify the safety of
%implementations of consensus and SMR algorithms in the EventML
%programming language. EventML interfaces with the Nuprl theorem
%prover, in which proofs are conducted, and uses Nuprl's code
%generation facilities.
%%The authors report proving a Paxos-like algorithm in a matter of
%%days.  One notable feature of EventML is that its semantics is based
%%on partial order between events, whereas most other tools employ the
%%standard model of Lamport, in which an execution is a total order on
%%events.
%
%Other works applied interactive theorem proving to verify Paxos
%protocols at the algorithmic level, without an executable
%implementation. \cite{DiskPaxos-AFP} proved the correctness of the
%Disk Paxos algorithm in Isabelle/HOL~\cite{nipkow_isabelle/hol:_2002},
%in about 6,500 lines of proof script. Recently,
%\cite{chand_formal_2016} presented safety proofs of Paxos and
%Multi-Paxos using the TLA+~\cite{lamport_temporal_1994} specification
%language and the TLA Proof System
%TLAPS~\cite{chaudhuri_tla+proof_2010}. TLA+ has also been used in
%Amazon to model check distributed algorithms~\cite{amazon}. However,
%they did not spend the effort required to obtain formal proofs, and
%only used the TLA+ models for bug finding via the TLA+ model
%checker~\cite{tlc}.
%
%Compared to our approach, using interactive theorem provers requires
%more user expertise and is more labor intensive. We note that
%part of the difficulty in using an interactive theorem prover lies in
%the unpredictability of the automated proof methods available and the
%considerable experience needed to write proofs in an idiomatic style
%that facilitates automation. An interesting direction of research is
%to integrate our methodology in an interactive theorem prover to
%achieve predictable automation in a style that is natural to systems
%designers.
%
%%% Compared to our approach, using interactive theorem provers requires
%%% more user expertise and is more labor intensive.  To create a more
%%% detailed comparison, we used Isabelle/HOL to prove the safety of a
%%% Paxos model which is analogous to the first-order logic model of Paxos
%%% of \Cref{fig:paxos-fol}. We report on this comparison in
%%% \commentout{\Cref{sec:isabelle}}\cite{oopsla17-epr-tr}. We note that
%%% part of the difficulty in using an interactive theorem prover lies in
%%% the unpredictability of the automated proof methods available and the
%%% considerable experience needed to write proofs in an idiomatic style
%%% that facilitates automation. An interesting direction of research is
%%% to integrate our methodology in an interactive theorem prover to
%%% achieve predictable automation in a style that is natural to systems
%%% designers.
%
%\paragraph{Works based on EPR}
%
%\cite{ivy} and \cite{ken_fmcad16} have also used EPR to verify
%distributed protocols and cache coherence protocols. \cite{ivy}
%develops an interactive technique for assisting the user to find
%universally quantified invariants (without quantifier alternations).
%In contrast, here we use invariants that contain quantifier alternations,
%as used in proofs of Paxos protocols.
%\cite{ken_fmcad16} goes beyond our work by extracting executable code from the modeling language.
%In the future, we plan to apply a similar extraction methodology to Paxos protocols.
%
%In \cite{CAV:IBINS13,POPL:IBILNS14,shachar_thesis} it was shown that
%EPR can express a limited form of transitive closure, in the context
%of linked lists manipulations. We notice that in the context of our methodology,
%%this work bears some similarity with our notion of  a derived relation, as
%their treatment of transitive closure can be considered
%as adding a derived relation. This work and our work both show that EPR is
%surprisingly powerful, when augmented with derived relations.
%
%In \cite{tacas17-bh}, bounded quantifier instantiation is explored as
%a possible solution to the undecidability caused by %that arises due to
%quantifier alternations. This work shares some of the motivation and
%challenges with our work, but proposes an alternative solution. The
%context we consider here is also wider, since we deal not only with
%quantifier alternations in the inductive invariant, but also with
%quantifier alternations in the transition relation. \cite{tacas17-bh}
%also shows an interesting connection between derived relations and quantifier instantiation, and these insights may
%apply to our methodology as well. An appealing future research direction is to
%combine user provided derived relations and rewrites
%together with heuristically generated quantifier instantiation.

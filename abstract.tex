\begin{abstract}

\commentout{
This paper addresses the problem of verifying the correctness of realistic distributed systems, all the way from
protocol design to efficient system implementation.
\gl{implementations}\oded{is this better?}
It shows how to harness modularity for simplifying the verification of such systems.
Rather then directly reasoning about existing code,
we present a programming language which both permits effective
reasoning and generation of efficient executable code.

The simplification is achieved by the usage of decidable logic to automatically check the correctness of straightline programs.
Unlike existing approaches, the usage of decidable logic guarantees that the solver always terminates with a proof or a concrete counterexample.
However, decidable logics are not a panacea. They are rather limited and common wisdom suggests that it is very hard to employ decidable reasoning to reason
about real systems.

This paper shows that existing modularity principles can be used to break the verification task into several subtasks that can be solved
via different decidable logics.
We embed this modularity in a real verification system and demonstrate the usefulness of the methodology
to verify realistic protocols, such as Raft, all the way from the design to an efficient implementation.
We also show that the verification effort is drastically smaller than existing approaches.

There are several reasons why this can be done for distributed systems:
(i)~The existence of a rather simple design of the protocol expressible in a rather limited language,
(ii)~The separation between abstract and concrete data structures, and
(iii)~The message passing communication model which makes it possible to treat each message handler as executing atomically.
}

Proof automation can substantially increase productivity in formal
verification of complex systems.  However, unpredictablility of automated provers in handling quantified formulas presents a major
hurdle to usability of these tools. We propose to solve this
problem not by improving the provers, but by using a modular proof
methodology that allows us to produce \emph{decidable} verification
conditions.  Decidability greatly improves predictability of proof
automation, resulting in a more practical verification approach. 
We apply this methodology to develop verified
implementations of distributed protocols, demonstrating its effectiveness.

%We
%demonstrate this by using the methodology to develop verified
%implementations of distributed protocols with competitive performance.


%\item The isolation between the local states at every node.

%\begin{inparaenum}[(i)]
%\item The existence of a rather simple design of the protocol expressible in a rather limited language,
%\item The separation between abstract and concrete data structures, and
%\item The message passing communication model which makes it possible to treat each message handler as executing atomically.
%%\item The isolation between the local states at every node.
%\end{inapraenum}


%
%
%
% \oded{I just put this here for us, we should write a real
%abstract} We consider challenging distributed systems, e.g., Raft,
%Muti-Paxos, Sharded Hash Table. We develop a methodology to
%strucutre both the systems implementation and its proof such that
%verification conditions are in decidable logics. We show that this
%resutls in shorted and more intuitive proofs, as well as predictable
%automation of VC checks.
\end{abstract}

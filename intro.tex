\section{Introduction}


%% Motivation
Distributed protocols play an essential role in important emerging technologies such as blockchains and autonomous vehicles.
%in these technologies.
Testing such systems is not always effective, and the cost of failures can be large.
This is well-recognized, e.g., in the ISO 26262 Functional Safety Standard \cite{ISO26262}.
Therefore, the systems community is increasingly receptive to formal methods \cite{Verdi,IronFleet,DBLP:conf/sosp/ChenZCCKZ15,DBLP:conf/sosp/NelsonSZJBTW17,DBLP:conf/sosp/ChenCKWICKZ17,DBLP:conf/sosp/FerraiuoloBHP17}.
However, existing verification techniques are difficult to apply
and require a large proof engineering effort. For example, the Verdi proof of the Raft protocol~\cite{raft} includes 50000 lines of proof for 530 lines of
implementation~\cite{VerdiCPP}, which required several person-years.

%SMT and their limitation.
Automated provers, such as the Z3 SMT solver~\cite{Z3} are capable of
greatly reducing the verification effort, as seen, e.g., in the
IronFleet project~\cite{IronFleet}.  By annotating the program with
invariants, the programmer reduces the proof to lemmas called
\emph{verification conditions} that can be discharged by the automated
prover.  In case these lemmas fail, the prover can sometimes produce
counterexamples that explain the failure and allow the
programmer to correct the annotations.

Unfortunately, the behavior of provers can be quite unpredictable,
especially when applied to formulas with quantifiers, which are common
when verifying distributed protocols. Since the problem presented to
the prover is in general undecidable, it is no surprise that the
prover sometimes diverges or produces inconclusive results on small
instances, or suffers from the ``butterfly effect'', when a seemly
irrelevant change in the input causes the prover to fail. As observed
in the IronFleet project, SMT solvers can fail even on tiny
examples. When this happens, the user has little information with
which to decide how to proceed.  This was identified in IronFleet as
the main hurdle in the verification task.

\paragraph{Decidable Reasoning} %Decidable Verification

We address this problem here by showing how to verify the safety of realistic
implementations of distributed systems using only \emph{decidable} verification
conditions. This is accomplished in part by an appropriate choice of
language to express the program, as in previous work~\cite{Ivy, dragoi_psync:_2016}, but also by using
modular reasoning techniques to structure the correctness proof such that
difference components can be proved using decidable fragments of first-order logic with various background theories.
%We will refer to this approach as \emph{theory segregation}\gl{segregation has negative connotation}.
We will refer to this approach as \emph{decidable decomposition}.
%\gl{segregation has negative connotation}.

\commentout{
separate the correctness proof into components
that can be proved using decidable fragments of first-order logic with
various background theories.
%We will refer to this approach as \emph{theory segregation}\gl{segregation has negative connotation}.
}

Because our prover is a decision procedure
for the logical fragments we use, we can guarantee that in principle
it will always terminate with a proof or a counter-model. In
practice, decidability means that the behavior of the prover is much
more predictable~\cite{oopsla17-epr}.
%We will observe that the additional
%human effort needed to achieve this result is modest, and that we
%can develop and verify realistic implementations of distributed
We observe in our evaluation that the
human effort needed to achieve %theory segregation
decidable decomposition is modest, and that we
can develop and verify realistic implementations of distributed
protocols without relying on the automated prover to solve undecidable
problems.

A key aspect of our methodology is the separation between the abstract
model and its\sharon{ removed "concurrent"} implementation using concrete data
structures. We prove system-level properties of the protocol using the
abstract model and use these properties as a lemma in proving the
implementation. This is important for obtaining stratification in the use of quantifiers.
In addition, we use abstract datatypes to hide the
theory needed to implement a data structure behind a set of abstract
properties. The resulting %theory segregation
decidable decomposition allows us to prove each
module locally with decidable logics, even though a global proof would take us
outside the decidable range.


\commentout{
This paper shows how to simplify the verification of realistic distributed systems using decidable logic.
The main idea is that the verification system enforces conditions on the program and the specification which guarantee that
the generated verification conditions are expressible in a decidable logic, i.e., there exists a terminating algorithm
which either proves the verification conditions or generates a counterexample.
Specifically, our system guarantees that SMT solvers must terminate with conclusive answers when checking these verification conditions.
This limits the class of programs and specifications.
However, we demonstrate that these limitations are reasonable for distributed systems.
In practice, this significantly reduces the complexity of the
verification, and our experience shows that in many cases an
undecidable logical formulation of a system and its proof can be simulated using multiple different decidable logics
(e.g., pure uninterpreted first-order logic without quantifier alternations, and decidable fragments based on theories).
%(pure first-order logic without quantifier alternation and quantifier free decidable theories)
For example, the verification of Raft in our system took $X$ lines of code and proof and $0.25$ person years.
This is in contrast to the Verdi project.
%which includes 50000 lines of proof for 530 lines of
%implementation~\cite{VerdiCPP}, which required several human years.

\paragraph{Modularity for Decidability}
Reasoning about distributed systems in decidable logics is hard since the one needs to account
for several features including: unbounded nodes, set cardinalities (e.g., majority), and arithmetics.
The combinations of these features are beyond the reach of existing decidable logics.
Indeed, this is where SMT tools fail.

This paper shows that existing modularity principles e.g., \cite{Larch} can be used to guarantee that verification
conditions are expressed in decidable logics.
The idea is that different parts of the code and proof actually need different decidable logics.
% and the system can actually figure this based on on their types.
Technically, the system deploys assume/guarantee reasoning where proofs are broken into several parts (modules) expressed
in different logics.

Decidability is enabled by the well-known separation between the abstract interface of a data types and its concrete implementation.
Thus, the concrete implementation can be verified to satisfy its interface in a suitable theory,
while code using the data type can be verified against the abstract interface, which can often be specified in uninterpreted first-order logic.
However, this does not usually suffice to obtain decidability.

A key enabling feature of proof modularity in distributed systems is the existence of an abstract design which omits many of
the implementation details.
This design is usually similar to the published article and abstracts some details of the actual code.
%which our system automatically generates.
Conceptually the design can be viewed as ``a lemma in the implementation'', which is proved separately.
Technically, the design implements abstract operations.
The implementation code uses these abstract operations as ghost code, and is verified in an assume/guarantee fashion by (i) verifying that before every abstract operation,
its precondition is met and assuming that after the operation its postcondition holds.

Notice that since we are working in decidable logic, this process is also useful to identify bugs in the designs and
the implementation.
When failed, the system generates counterexamples and error messages for bugs in the implementation, the design, and the
interface specification.


\paragraph{Generating Executable Code}
}

At the end of the process, we compile the verified system to executable code (which uses a small set of trusted libraries, e.g., to implement a built-in array type).
%Built-in language features are implemented as part of the trusted code based.  by connecting several trusted libraries.
Our preliminary experience indicates that this can be used to generate reasonably efficient verified distributed systems.


\paragraph{Contributions}
The contributions of our paper are: % as follows:
\begin{compactenum}
%\item A new methodology for decomposing the safety verification of realistic distributed systems, from protocol to implementation, into decidable sub-problems.
\item A new methodology, \emph{decidable decomposition}, for verifying realistic distributed systems
  using existing modularity principles to decompose the verification into lemmas proved in (different) decidable logics.
\item Verified implementations of two popular distributed algorithms, Raft and Multi-Paxos,
  showing that proofs of practical systems naturally decompose into decidable sub-problems.
%by decomposing the problem into decidable sub-problems.
\item An evaluation of the verification effort and run-time performance of these systems,
  showing that the effort required by our methodology is an order of magnitude smaller than with previous methodologies such as Verdi or IronFleet, and the obtained performance is comparable.
\end{compactenum}
\commentout{
\begin{compactenum}
\item A new methodology for verifying distributed systems from design to efficient implementation,
based on modular reasoning and SMT solvers. % SMT solvers.
\item Showing that existing modular proof systems can be used to
compositionally reason about real systems using decidable logics.
\commentout {
Here, the programmer decides the modularity boundaries and the interfaces and the
system automatically verifies the correctness or identifies errors or interface
mismatch.}
\item Showing that the methodology is effective to
obtain efficient verified implementations of protocols such as Raft with reasonable human effort and reliable prover performance.
\end{compactenum}
}

\ken{Added the following paragraph in response to first shepherd
  request.}  The practical results in this paper are based on a tool
called IVy. IVy was introduced in~\cite{Ivy} as a tool for interactive
verification of distributed protocols using the decidable logic
EPR. IVy's modular proof system was introduced
in~\cite{ken_fmcad16}. We use that system here (with some extensions
to the logic) to demonstrate a new methodology of decidability through
modularity. This methodology allows us to formally refine abstract
distributed protocol models from~\cite{oopsla17-epr} into executable
implementations, while keeping the verification conditions decidable.
